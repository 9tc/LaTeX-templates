\documentclass[a4j,10pt]{jarticle}

\title{タイトル}
\author{dokaraya}
\date{\today}

\usepackage{fancyheadings}
\usepackage[dvipdfmx]{graphicx}
\usepackage[dvipdfmx]{color}
\usepackage{amsmath}

% スタイルファイル
\usepackage{siunitx}
\usepackage{bm}
\usepackage{listings}
\usepackage{comment}
\usepackage{algorithm}
\usepackage[noend]{algorithmic}
\usepackage{url}
\usepackage{here}
\renewcommand{\lstlistingname}{ソースコード}
\lstset{
    frame=single,
    numbers=left,
    tabsize=2,
    language=c++,
    basicstyle={\scriptsize}
}

\renewcommand{\algorithmicrequire}{\textbf{Input:}}
\renewcommand{\algorithmicensure}{\textbf{Output:}}

\makeatletter
\renewcommand{\ALG@name}{アルゴリズム}
\makeatother

\renewcommand{\labelenumi}{\arabic{enumi}.}
\renewcommand{\labelenumii}{\arabic{enumi}.\arabic{enumii}.}
\renewcommand{\labelenumiii}{\arabic{enumi}.\arabic{enumii}.\arabic{enumiii}.}

\begin{document}
	\maketitle
  \section{内容}
  犬と猫はかわいいことが知られている.ソースコード\ref{lis:yay}に示すプログラムを用いて,犬と猫の反応を観察した.その反応の例を図\ref{fig:cat}に示す.\par
  これらの反応を200人に見せてどちらがかわいいか選んでもらった.その結果を表\ref{tab:example}に示す.よってどちらもかわいいことがわかった.

  \begin{lstlisting}[caption = Yayするプログラム, label = lis:yay]
  fun main(){
    print("Yay!")
  }
  \end{lstlisting}

  \begin{figure}
    \centering
    \includegraphics[width=9cm]{fig/img.jpg}
    \caption{素晴らしき猫}
    \label{fig:cat}
  \end{figure}

  \begin{table}[t]
    \caption{かわいさ投票の結果}
    \label{tab:example}
    \centering
    \begin{tabular}{cr}
      \hline
      種類 & かわいさ(票)\\ \hline
      犬 & 100\\
      猫 & 100\\ \hline
      総数 & 200 \\ \hline
    \end{tabular}
  \end{table}
\end{document}
